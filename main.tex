\documentclass{beamer}
%%%%%%%%%%%%%%%%%%%%%%%%%%%%%
% Do not change the next few lines. 
\usepackage{graphicx, amsfonts, amssymb, amsthm,xcolor,subfigure, multicol,multimedia,breqn}
\newcommand{\singleslidebig}[1]{\begin{frame}
\begin{center}
\textcolor{red}{\Huge{#1}}
\end{center}
\end{frame}}
\usepackage{setspace}

\usepackage{pifont}
%%%%%%%%%%%%%%%%%%%%%%%%%%%%%

\newcommand{\beq}{\begin{equation}}
\newcommand{\eeq}{\end{equation}}
\newcommand{\al}{\alpha}
\newcommand{\la}{\lambda}
\newcommand{\va}{\varphi}
\newcommand{\vare}{\varepsilon}
\newcommand{\pa}{\partial}
\newcommand{\tri}{\triangle}
\newcommand{\trid}{\triangledown}
%\newcommand{\proof}{{\textbf{Proof} \ \ }}
\newcommand{\eproof}{$\quad \Box$}
\newcommand{\bea}{\begin{eqnarray}}
\newcommand{\eea}{\end{eqnarray}}
\newcommand{\bean}{\begin{eqnarray*}}
\newcommand{\eean}{\end{eqnarray*}}
\newcommand{\bx}{{\bf x}}
\newcommand{\bn}{{\bf n}}
\newcommand{\bd}{{\bf d}}
\newcommand{\bk}{{\bf k}}
\newcommand{\cM}{{\mathcal{M}}}
\newcommand{\ep}{{\epsilon}}
\newcommand{\aeps}{^{\epsilon}_a}
\newcommand\XOR{\mathbin{\char`\^}}
\usetheme{Warsaw}
\newtheorem{cor}{Corollary}
\newtheorem{prop}{Proposition}

%\setbeamercolor{item projected}{bg=darkred}
\setbeamertemplate{enumerate items}[default]
%\setbeamertemplate{navigation symbols}{}
%\setbeamercolor{block title}{fg=darkred}
%\setbeamercolor{local structure}{fg=darkred}
% \usepackage{beamerthemesplit} // Activate for custom appearance
\definecolor{myor}{RGB}{245,102,0}
\definecolor{mypur}{RGB}{82,45,128}



\setbeamercolor{structure}{fg=myor!90!mypur}

\newtheorem*{model}{\footnotesize{Hodgkin-Huxley Model}}

%\setbeamercolor{alerted text}{fg=red}
%%%%%%%%%%%%%%%%%%%%%%%%%%%%%
% Now Start entering your information.
\title[Perturbation of Semigroups]{Perturbation of Semigroups}
\author[Manning and Spiess]{\textbf{Jacob Manning and Cameron Spiess}}
\institute[Clemson University]{Clemson University}
\date{December 5, 2024}
\begin{document}

\maketitle

\begin{frame}{3.1 Perturbations by Bounded Linear Operators}
    \bf{\underline{Section 3.1 Perturbations by Bounded Linear Operators}}
\end{frame}

\begin{frame}{3.1 Perturbations by Bounded Linear Operators}
    \begin{definition}
        \begin{itemize}
            \item If $A$ is the infinitesimal generator of $T(t)$ (a semigroup of linear operators) then we say $A$ generates $\{T(t): t\geq 0\}$.
            \item $C_0 \text{ semigroup } T(t)$ is the semigroup of strongly continuous operators that are bounded on $X$.
        \end{itemize}
    \end{definition}
    \begin{theorem}[3.1.1]
        Let $X$ be a Banach space and let $A$ be the infinitesimal generator of a $C_0$ semigroup $T(t)$ on $X$, satisfying $||T(t)||\leq Me^{\omega t}$. If $B$ is a bounded linear operator on $X$ then $A+B$ is the infinitesimal generator of a $C_0$ semigroup $S(t)$ on $X$, satisfying $||S(t)||\leq Me^{(\omega+M||B||)t}$.
    \end{theorem}
\end{frame}

\begin{frame}{3.1 Perturbations by Bounder Linear Operators}
    \textbf{\underline{Remark}}\\
    \begin{itemize}
        \item An equivalent definition of infinitesimal generator of $\{T(t): t\geq 0$ is, let $A:D(A)\subset X\rightarrow X$ is defined by $D(A)=\{x\in X: \lim_{t\downarrow 0}\frac{1}{t}(T(t)x-x) \text{ exists}\}$ and $Ax= \lim_{t\downarrow 0}\frac{1}{t}(T(t)x-x)$
        \item The initial bound of $||T(t)||\leq Me^{\omega t}$ is given to the true generator $A$. However, there will exist some error when computing this generating numerically. The numerically computed generator could be $\tilde{A}=A+\Delta A$ where $\Delta A$ is a bounder linear operator. 
    \end{itemize}
\end{frame}

\begin{frame}{3.1 Perturbations by Bounded Linear Operators}
We are interested in relations between the semigroup $T(t)$ generated by $A$ and the semigroup $S(t)$ generated by $A+B$.
Consider the following operator,
$$H(s)=T(t-s)S(s)$$ for $x\in D(A)=D(A+B)$ such that $s\rightarrow H(s)x$. Notice that $H(s)$ is differentiable. 
$$H'(s)x=T(t-s)BS(s)x\text{ for }x\in D(A)$$
Integrating this from $0$ to $t$, we can find that 
$$S(t)x=T(t)x+\int_0^tT(t-s)BS(s)xds$$
Both sides are bounded, meaning $S(t)$ is a solution. 
\end{frame}

\begin{frame}{3.1 Perturbations by Bounded Linear Operators}
    \begin{theorem}[3.1.2]
        Let $T(t)$ be a $C_0$ semigroup satisfying $||T(t)||\leq Me^{\omega t}$. Let $B$ be a bounded operator on $X$. Then there exists a unique family $V(t)$, $t\geq 0$ of bounded operators on $X$ such that $t\to V(t)x$ is continuous on $[0,\infty)$ for every $x\in X$ and
        \begin{align*}
            V(t)x=T(t)x+\int_0^t T(t-s)BV(s)xds\qquad x\in X.
        \end{align*}
    \end{theorem}
    \textbf{\underline{Remark}}\\
    We can immediately gain an explicit representation of $S(t)$ in terms of $T(t)$ such that $$S(t)=\sum_{n\in\mathbb{N}} S_{n-1}(t)$$ where $S_0(t)=T(t)$ and $S_{n}(t)=\int_0^t T(t-s)BS_{n-1}(s)xds$ with $x\in X$. The series converges uniformly.
\end{frame}

\begin{frame}{3.1 Perturbations by Bounded Linear Operators}
    \begin{cor} [3.1.3]
        Let $A$ be the infinitesimal generator of a $C_0$ semigroup $T(t)$ satisfying $||T(t)||\leq Me^{\omega t}$. Let $B$ be a bounded operator and let $S(t)$ be the $C_0$ semigroup generated by $A+B$. Then
        \begin{align*}
            ||S(t)-T(t)||\leq Me^{\omega t}(e^{M||B||t}-1).
        \end{align*}
    \end{cor}\vspace{-1mm}
    \begin{proof}
        From our theorems 3.1.1 and 3.1.2 we gain the following
        $$ ||S(t)x-T(t)x||\leq \int_0^t ||T(t-s)||\text{ }||B||\text{ }||S(s)||\text{ }||x||ds$$
        $$\leq M^2e^{\omega t}||B||\int_0^t e^{M||B||s}||x||ds$$
        $$=Me^{\omega t}(e^{M||B||t}-1)||x||$$
    \end{proof}
\end{frame}

\begin{frame}{3.1 Perturbations by Bounded Linear Operators}
    \textbf{\underline{Remark}}\\
    The big takeaway of this section is theorem 3.1.1, which shows that we can add a bounded linear operator to an infinitesimal operator and it does not destroy the property. \\\vspace{5mm}
    \textbf{\underline{Follow up question}}\\
    What other properties can a semigroup $T(t)$ generated by an infinitesimal operator $A$ carry over?

    \vspace{5mm}
        It turns our perturbation ($A+B$) will be compact or analytic if our infinitesimal operator $A$ is compact or analytic. 
\end{frame}

\begin{frame}{3.1 Perturbations by Bounded Linear Operators}
    \begin{prop}[3.1.4]
        Let $A$ be the infinitesimal generator of a compact $C_0$ semigroup $T(t)$. Let $B$ be a bounded operator, then $A+B$ is the infinitesimal generator of a compact $C_0$ semigroup $S(t)$.
    \end{prop}
    \begin{prop}[3.2.1]
        Let $A$ be the infinitesimal generator of an analytic semigroup. Let $B$ be a closed linear operator satisfying $D(B)\supset D(A)$ and 
        $$||Bx||\leq a||Ax||+b||x||$$ for $x\in D(A)$
        There exists a positive number $\delta$ such that if $0\leq a\leq \delta$ then $A+B$ is the infinitesimal generator of an analytic semigroup
    \end{prop}
    
\end{frame}

\begin{frame}{3.1 Perturbations by Bounded Linear Operators}
    \textbf{\underline{Remark}}\\
    Not all of the properties of the semigroup of $T(t)$ are preserved by a bounded perturbation of its infinitesimal generator. \vspace{5mm}

    \textbf{\underline{Ex}}\\
    Let $A$ be the infinitesimal generator of a semigroup $T(t)$ which is continuous in the uniform operator topology for $t\geq t_0\geq 0$, or is differentiable for $t\geq t_0\geq 0$, or is compact for $t\geq t_0\geq 0$. 
    Then $S(t)$, the semigroup generated by $A+B$ where $B$ is a bounded operator need not have the corresponding property. 
    
    
\end{frame}

\begin{frame}{3.3 Perturbations of Infinitesimal Generators of Contraction Semigroups}
    \bf{\underline{3.3 Perturbations of Infinitesimal Generators}}
    \bf{\underline{of Contraction Semigroups}}
\end{frame}

\begin{frame}{3.3 Perturbations of Infinitesimal Generators of Contraction Semigroups}
    Recall
    \begin{theorem}[1.4.3 Lumer-Phillips]
        \begin{itemize}
            \item If $A$ is dissipative and there is a $\lambda _0 >0$ such that $R(\lambda _0 I-A)=X$, then $A$ is the infinitesimal generator of a $C_0$ semigroup of contractions on $X$.
            \item If $A$ is the infinitesimal generator of a $C_0$ semigroup of contractions on $X$ then $R(\lambda I-A)=X$ for all $\lambda >0$ and A is dissipative.
        \end{itemize}
    \end{theorem}
\end{frame}

\begin{frame}{3.3 Perturbations of Infinitesimal Generators of Contraction Semigroups}
    \begin{definition}[3.3.1]
        \begin{itemize}
            \item A linear operator $A$ is called dissipative if $Re\langle Ax,x\rangle <0\quad \forall x\in D(A)$
            \item A dissipative operator $A$ such that $R(I-A)=X$ is called a maximal dissipative operator.
            \item An operator $A$ is called closable if there exists a closed operator $\bar{A}$ such that $G(A)\subset G(\bar{A})$.
        \end{itemize}
    \end{definition}
    \textbf{\underline{Remark}}\\
    If $A$ is dissipative, for any $c>0$, $cA$ is also dissipative.\\
    \begin{theorem}[Lumer-Phillips]
        A densely defined operator $A$ is the infinitesimal generator a $C_0$ semigroup of contractions iff $A$ is maximal dissipative.
    \end{theorem}
\end{frame}

\begin{frame}{3.3 Perturbations of Infinitesimal Generators of Contraction Semigroups}
    \begin{theorem}[3.3.2]
        Let $A$ and $B$ be linear operators in $X$ such that $D(B)\supset D(A)$ and $A+tB$ is dissipative for $0\leq t\leq 1$. If
        \begin{align*}
            ||Bx||\leq \alpha ||Ax||+\beta ||x|| \qquad x\in D(A)
        \end{align*}
        where $0\leq \alpha <1$, $\beta \geq 0$ and for some $0\leq t_0\leq 1$, $A+t_0 B$ is maximal dissipative then $A+tB$ is maximal dissipative for all $0\leq t\leq 1$.
    \end{theorem}
    This theorem is often used with the following corollary.
\end{frame}

\begin{frame}{3.3 Perturbations of Infinitesimal Generators of Contraction Semigroups}
    \begin{cor}[3.3.3]
        Let $A$ be the infinitesimal generator of a $C_0$ semigroup of contractions. Let $B$ be dissipative and satisfy $D(B)\supset D(A)$ and
        \begin{align*}
            ||Bx||\leq \alpha ||Ax||+\beta ||x|| \qquad x\in D(A)
        \end{align*}
        where $0\leq \alpha <1$, $\beta \geq 0$. Then $A+B$ is the infinitesimal generator of a $C_0$ semigroup of contractions.
    \end{cor}
    \textbf{\underline{Remark}}\\
        If $A+tB$ is dissipative for $0\leq t\leq 1$, $D(B)\supset D(A)$, $\overline{D(A)}=X$, and 
        \begin{align*}
            ||Bx||\leq \alpha ||Ax||+\beta ||x|| \qquad x\in D(A)
        \end{align*}
        where $0\leq \alpha <1$, $\beta \geq 0$. Then either both $A$ and $A+B$ are maximal dissipative or neither.
\end{frame}

\begin{frame}{3.3 Perturbations of Infinitesimal Generators of Contraction Semigroups}
    $\alpha =1$ can be an issue since $A+B$ may not be closed and thus cannot be the infinitesimal generator of a $C_0$ semigroup.\\
    \pause
    \textbf{\underline{Ex}}\\
    Let $iA$ be a self adjoint operator in a Hilbert space. It follows that $A$ and $-A$ are infinitesimal generators of $C_0$ semigroups of contractions. Taking $\alpha =1$ and $\beta =0$. It follows
    \begin{align*}
        ||-Ax||\leq 1*||Ax||+0*||x||=||Ax||
    \end{align*}
    However $\overline{A+(-A)}\Big |_{D(A)}$ is not closed.
\end{frame}

\begin{frame}{3.3 Perturbations of Infinitesimal Generators of Contraction Semigroups}
    What conditions can we set $\alpha =1$?
    \pause
    \begin{theorem}[3.3.4]
        Let $A$ be the infinitesimal generator of a $C_0$ semigroup of contractions. Let $B$ be dissipative and satisfy $D(B)\supset D(A)$ and
        \begin{align*}
            ||Bx||\leq ||Ax||+\beta ||x|| \qquad x\in D(A)
        \end{align*}
        where $\beta \geq 0$. If $B^*$ is densely defined, then the closure $\overline{A+B}$ of $A+B$ is the infinitesimal generator of a $C_0$ semigroup of contractions.
    \end{theorem}
\end{frame}

\begin{frame}{3.3 Perturbations of Infinitesimal Generators of Contraction Semigroups}
    If $X$ is a reflexive Banach space and $T$ is closable and densely defined, we know that $T^*$ is closed and $D(T^*)$ is dense in $X^*$.\\
    It follows
    \begin{cor}[3.3.5]
        Let $X$ be a reflexive Banach space and let $A$ be the infinitesimal generator of a $C_0$ semigroup of contractions in $X$. Let $B$ be dissipative such that $D(B)\supset D(A)$ and
        \begin{align*}
            ||Bx||\leq ||Ax||+\beta ||x|| \qquad x\in D(A)
        \end{align*}
        where $\beta \geq 0$. Then $\overline{A+B}$, the closure of $A+B$, is the infinitesimal generator of a $C_0$ semigroup of contractions in $X$.
    \end{cor}
\end{frame}

\begin{frame}{Citations}
    Amnon Pazy. "Perturbations and Approximations." Semigroups of Linear Operators and Applications to Partial Differential Equations. Springer Science \& Business Media, 6 Dec. 2012.
\end{frame}

\end{document}
